%%*** Changes.tex *************************************************************
%%
%% This file is part of BibTool.
%% It is distributed under the Creative Commons Attribution-Share
%% Alike 3.0 License.
%%
%% (c) 2010-2019 Gerd Neugebauer
%%
%% Net: gene@gerd-neugebauer.de
%%
%%-----------------------------------------------------------------------------
%% Usage:  latex     Changes
%%         latex     Changes
%%*****************************************************************************
\documentclass[11pt,a4paper]{scrartcl}

\usepackage{multicol}
\usepackage{ragged2e}

\newenvironment{Developers}{}{}
\newcommand\Developer[3]{}
\newcommand\Arg[1]{\texttt{#1}}
\newcommand\rsc[1]{\texttt{#1}}
\newcommand\File[1]{\textsf{#1}}
\newcommand\BibTool{\textsc{BibTool}}
\newcommand\BibTcl{\textsc{BibTcl}}
\newcommand\BibTeX{\textsc{Bib}\TeX}
\newenvironment{Release}[2]{%
  \def\tmp{#2}%
  \section*{Release #1 \ifx\tmp\empty\else{\normalsize[#2]}\fi}
  \begin{itemize}
}{\end{itemize}}
\newenvironment{Fix}[1]{\item }{}
\newenvironment{New}[1]{\item }{}
\newenvironment{Doc}[1]{\item }{}
\newenvironment{Update}[1]{\item }{}
\newcommand\BS{\char`\\}

\begin{document}%%%%%%%%%%%%%%%%%%%%%%%%%%%%%%%%%%%%%%%%%%%%%%%%%%%%%%%%%%%%%%%

\title{\BibTool{} Change Log}
\author{Gerd Neugebauer}
\date{}
\maketitle

\begin{Developers}
  \Developer{gene}{Gerd Neugebauer}{gene@gerd-neugebauer.de}
\end{Developers}

\begin{multicols}2\RaggedRight

 % =====================================================================
 \begin{Release}{2.68}{Janary ??, 2019}
  \begin{Fix}{gene}
    Nasty bug in the memory management fixed which could have appeared
    during extracting by aux file.
  \end{Fix}
  \begin{Update}{gene}
    Handling of extracting by aux file improved.
  \end{Update}
  \begin{New}{gene}
    Record extended by attribute \texttt{lineno}.
    This attribute carries the line number of the initiating \@.
  \end{New}
  \begin{Update}{gene}
    The format of the error messages has been slightly streamlined.
    Double spaces are avoided; two colons in one message are avoided.
  \end{Update}
  \begin{Update}{gene}
    The format of the messages of check_rule include file and line number.
  \end{Update}
  \begin{New}{gene}
    Warning for double fields added.
  \end{New}
  \begin{New}{gene}
    The empty output file is used to signal that the output should be
    suppressed.
  \end{New}
  \begin{New}{gene}
    The resources \rsc{check.warning.rule} and \rsc{check.error.rule}
    have been introduced to allow semantic checks to be classified as
    warning or error.
  \end{New}
  \begin{New}{gene}
    The behaviour of the resource \rsc{check.double} has been
    generalized. The requirement that double entries to be adjacent
    has been dropped. This has the impact that the processing is
    slightly slower.
  \end{New}
 \end{Release}

 % =====================================================================
 \begin{Release}{2.67}{January 14, 2017}
  \begin{Fix}{gene}
    Treatment of empty values for key separators fixed.
  \end{Fix}
  \begin{New}{gene}
    Resource \rsc{parse.exit.on.error} added to force a stop when a parsing
    error is encountered. The default is false, i.e. an error recovery is
    attempted.
  \end{New}
  \begin{Update}{gene}
    Error messages for parsing resource arguments improved. The usual pointer
    is used there as well.
  \end{Update}
  \begin{Update}{gene}
    Documentation extended to cover the exit codes of \BibTool.
  \end{Update}
  \begin{Update}{gene}
    Some updates in the documentation files for the new \TeX\ live:
    removing old font commands.
  \end{Update}
 \end{Release}

 % =====================================================================
 \begin{Release}{2.66}{October 30, 2016}
  \begin{Fix}{gene}
    \File{lib/biblatex.rsc} fixed.
  \end{Fix}
  \begin{New}{gene}
    Resource \rsc{keep.field} added to get rid of unknown fields.
  \end{New}
  \begin{Update}{gene}
    Internal memory management rewritten to get rid of unused memory
    some time in the future. For now just the C code is changed
    incompatibly if the macro \verb|COMPLEX_SYMBOL| is defined upon
    compilation.
  \end{Update}
  \begin{Update}{gene}
    Refactoring applied to make use of the C99 feature
    \File{stdbool.h}.
  \end{Update}
 \end{Release}

 % =====================================================================
 \begin{Release}{2.65}{June 14, 2016}
  \begin{Fix}{gene}
    Hash function and value storing for field name mapping during xref
    expansion fixed.
  \end{Fix}
 \end{Release}

 % =====================================================================
 \begin{Release}{2.64}{June 6, 2016}
  \begin{New}{gene}
    New resource \rsc{expand.xdata} introduced to control the
    expansion of xdata references.
  \end{New}
  \begin{New}{gene}
    New resource \rsc{crossref.map} introduced to add a field name
    mapping for cross-refs.
  \end{New}
  \begin{New}{gene}
    New resource \rsc{clear.crossref.map} introduced to clear the
    field name mapping for cross-refs.
  \end{New}
  \begin{Update}{gene}
    \File{io.c} introduced to contain input and output file parameters.
  \end{Update}
  \begin{Update}{gene}
    Documentation switched to Lua\LaTeX.
  \end{Update}
 \end{Release}

 % =====================================================================
 \begin{Release}{2.63}{January 16, 2016}
  \begin{Fix}{gene}
    Omission of the previous release fixed.
  \end{Fix}
 \end{Release}

 % =====================================================================
 \begin{Release}{2.62}{January 16, 2016}
  \begin{Fix}{gene}
    Combined rewrite and delete bug fixed.
  \end{Fix}
  \begin{Update}{gene}
    Minor improvements for the distribution.
  \end{Update}
  \begin{Update}{gene}
    \File{README} renamed to \File{README.md} to comply with the
    conventions of the GitHub repository.
  \end{Update}
 \end{Release}

 % =====================================================================
 \begin{Release}{2.61}{July 12, 2015}
  \begin{New}{gene}
    New resource \rsc{rename.field} introduced to conditionally rename a
    field.
  \end{New}
  \begin{Update}{gene}
    Typedef \verb|String| for \verb|Uchar*| and used across.
  \end{Update}
  \begin{Fix}{gene}
    Duplicate file \File{doc/bibtool.tex} in the distribution tar ball
    eliminated.
  \end{Fix}
 \end{Release}

 % =====================================================================
 \begin{Release}{2.60}{June 9, 2015}
  \begin{Fix}{gene}
    The field name and other symbols may start with any allowed
    character. Non-alpha characters at the beginning are treated as
    warning only.
  \end{Fix}
 \end{Release}

 % =====================================================================
 \begin{Release}{2.59}{March 14, 2015}
  \begin{Fix}{gene}
    Fix in \File{print.c} to omit an empty line after overflowing lines.
  \end{Fix}
  \begin{Fix}{gene}
    Fix in \File{print.c} to avoid an overflow situation.
  \end{Fix}
 \end{Release}

 % =====================================================================
 \begin{Release}{2.58}{February 9, 2015}
  \begin{Update}{gene}
    Library \File{tex\_def.rsc} extended with the primitives \verb|\i| and
    \verb|\j|.
  \end{Update}
  \begin{Update}{gene}
    The source tar is signed. The key of the author can be obtained
    from \texttt{pgp.mit.edu}.
  \end{Update}
  \begin{Update}{gene}
    \texttt{Test} renamed to \texttt{test}.
  \end{Update}
 \end{Release}

 % =====================================================================
 \begin{Release}{2.57}{April 18, 2014}
  \begin{Fix}{gene}
    Segfault in deTeX fixed.
  \end{Fix}
  \begin{Fix}{gene}
    Compiler warnings silenced.
  \end{Fix}
 \end{Release}

 % =====================================================================
 \begin{Release}{2.56}{April 14, 2014}
  \begin{Fix}{gene}
    Disambiguating numbers adapted to fit to documentation.
  \end{Fix}
  \begin{Fix}{gene}
    Configuration of regex fixed to work on Linux.
  \end{Fix}
  \begin{Fix}{gene}
    Documentation typos fixed.
  \end{Fix}
  \begin{Fix}{gene}
    Signed characters fro translation tables changed to unsigned.
  \end{Fix}
  \begin{Fix}{gene}
    Autoconf configuration improved.
  \end{Fix}
 \end{Release}

 % =====================================================================
 \begin{Release}{2.55}{April 15, 2012}
  \begin{New}{gene}
    Library \File{biblatex.rsc} added. It contains capitalizations of
    fields used in bib\LaTeX.
  \end{New}
  \begin{Fix}{gene}
    Fix for a misbehavior when selecting entries according to an aux file
    with deeply nested @strings.
  \end{Fix}
 \end{Release}

 % =====================================================================
 \begin{Release}{2.54}{February 21, 2012}
  \begin{New}{gene}
    Command line parameter \Arg{-V} documented.
  \end{New}
  \begin{New}{gene}
    Resource \rsc{key.make.alias} added to create new \texttt{@ALIAS}
    records for any newly generated key.
  \end{New}
  \begin{New}{gene}
    Resource \rsc{apply.alias} added to expand the \texttt{@ALIAS}
    records.
  \end{New}
  \begin{New}{gene}
    Resource \rsc{apply.include} added to expand the \texttt{@INCLUDE}
    records.
  \end{New}
  \begin{New}{gene}
    Resource \rsc{apply.modify} added to expand the \texttt{@MODIFY}
    records.
  \end{New}
  \begin{Update}{gene}
    Error message for \Arg{-o} without parameter added.
  \end{Update}
  \begin{Fix}{gene}
    \texttt{@ALIAS} and \texttt{@INCLUDE} records where not printed.
    This has been fixed.
  \end{Fix}
  \begin{New}{gene}
    A test suite for \BibTool{} has been integrated. It requires Perl
    to be present on the system on which the tests should be run.
  \end{New}
 \end{Release}

 % =====================================================================
 \begin{Release}{2.53}{September 25, 2011}
  \begin{Update}{gene}
    In \texttt{tex.define} spaces before the = are ignored instead of
    leading to unwanted definitions.
  \end{Update}
  \begin{Fix}{gene}
    An initialization error showed up on MacOS. This has been fixed.
  \end{Fix}
  \begin{Fix}{gene}
    The prepared makefiles for various operating systems missed an
    entry for \texttt{crossref.[cho]}.
  \end{Fix}
  \begin{Fix}{gene}
    Typo in help text and copyright year fixed.
  \end{Fix}
 \end{Release}

 % =====================================================================
 \begin{Release}{2.52}{June 6, 2011}
  \begin{Fix}{gene}
    A few incompatibilities with signed and unsigned characters which
    showed up on MacOS only have been fixed.
  \end{Fix}
 % \end{Release}
 % =====================================================================
 % \begin{Release}{2.52}{April 26, 2011}
  \begin{Fix}{gene}
    A few compiler warnings have been fixed.
  \end{Fix}
 \end{Release}

 % =====================================================================
 \begin{Release}{2.51}{May 12, 2010}
  \begin{Fix}{gene}
    Bug fix in \File{names.c}: The classification of name parts
    erroneously has used a wrong translation under certain circumstances.
  \end{Fix}
 \end{Release}

 % =====================================================================
 \begin{Release}{2.50}{May 12, 2010}
  \begin{Fix}{gene}
    Bug fix in \File{names.c}: The classification of name parts
    erroneously has used a wrong translation under certain
    circumstances.
  \end{Fix}
 %\end{Release}
 % =====================================================================
 %\begin{Release}{2.50}{February 6, 2010}
  \begin{Fix}{gene}
    Bug fix in \File{key.c}: The classification of name parts
    erroneously has considered, to be a first name under certain
    circumstances. This has been improved.
  \end{Fix}
  \begin{Fix}{gene}
    Improvement in \File{names.c} and \File{key.c} concerning printing in
    debug mode.
  \end{Fix}
 \end{Release}

 % =====================================================================
 \begin{Release}{2.49}{February 8, 2007}
  \begin{New}{gene}
    New resource \rsc{expand.crossref} introduced. This resource
    enables the expansion of fields inherited via the cross-ref
    mechanism.
  \end{New}
  \begin{Fix}{gene}
    Bug fix in \File{names.c}: The classification of name parts
    erroneously has considered the initials V., I., and X. to be Roman
    numerals belonging to the jr part under special circumstances.
    This has been improved.
  \end{Fix}

 % =====================================================================
 %\begin{Release}{2.49}{January 5, 2010}
  \begin{New}{gene}
    Name formatting now uses a fractional part to restrict the number
    of name parts considered.
  \end{New}

 % =====================================================================
 %\begin{Release}{2.49}{February 8, 2007}
  \begin{New}{gene}
    New resource \rsc{expand.crossref} introduced. This resource
    enables the expansion of fields inherited via the cross-ref
    mechanism.
  \end{New}
  \begin{Fix}{gene}
    Bug fix in \File{print.c}: Some national characters have been lost.
    It was necessary to use an unsigned character as elsewhere.
  \end{Fix}
 \end{Release}

 % =====================================================================
 \begin{Release}{2.48}{August 6, 2004}
  \begin{New}{gene}
    Resource \rsc{print.terminal.comma} introduced. This resource can
    be used to request commas after the last record even if it
    contradicts the rules of \BibTeX.
  \end{New}
  \begin{New}{gene}
    \File{makefile.in} for configure extended to support the variable
    \texttt{INSTALLPREFIX} to be prepended to installation
    directories.
  \end{New}
  \begin{Fix}{gene}
    Bug fix in \File{main.c}: Some compilers didn't like the method
    true().
  \end{Fix}
  \begin{Fix}{gene}
    Bug fix for kpathsea under Suse 9.1.
  \end{Fix}
  \begin{Fix}{gene}
    Bug fixes in expansion.
  \end{Fix}
 %\end{Release}

 % =====================================================================
 %\begin{Release}{2.48}{February 8, 2004}
   \begin{Update}{gene}
     \BibTcl{} has been taken out of the default making since it has
     caused troubles on several machines.
   \end{Update}
   \begin{Update}{gene}
     The printing of Strings honors now the dependencies that
     \BibTeX{} relies on. This has been accomplished by a change in
     the printing routine for databases. No new resource is involved.
   \end{Update}
   \begin{Update}{gene}
     Minor modification to the C API.
   \end{Update}
  \begin{New}{gene}
    Feature: When searching for an aux file and none is found then
    \texttt{.aux} is appended and a reopen is tried with the new name.
  \end{New}
  \begin{Update}{gene}
    Feature: When \rsc{expand.macros} is set the macros will be
    suppressed automatically. No need to set \rsc{print.entry.types}
    any more.
  \end{Update}
  \begin{New}{gene}
    The boolean resource \rsc{select.crossrefs} has been introduced.
    This resource can be used to force cross-referenced entries for
    selected entries to be included in the output.
  \end{New}
  \begin{Update}{gene}
    Feature: deleted entries are not reported as "`written"' by the
    statistics any more.
  \end{Update}
  \begin{Fix}{gene}
    Bug fix: Bug in extracting by aux file. Crossrefed entries where
    sometimes ignored.
  \end{Fix}
  \begin{Fix}{gene}
    Bug fix: Bug in parsing of backslashes in resources has been fixed.
    Unfortunately this bug appeared in one of the examples in the
    manual.
  \end{Fix}
  \begin{Fix}{gene}
    Bug fix: Bug in expansion introduced in 2.47 has been fixed. An
    error occurred when expanding macros.
  \end{Fix}
  \begin{New}{gene}
    The resource loading with \rsc{resource} now issues a warning when
    the file could not be found.
  \end{New}
  \begin{Doc}{gene}
    The meaning of post for the format specifiers \%d and \%D has been
    documented.
  \end{Doc}
  \begin{Fix}{gene}
    Bug fix: \BibTool{} has misinterpreted pre for the format specifier
    \%t and \%T due to an over-optimization. This has been fixed.
  \end{Fix}
  \begin{Fix}{gene}
    Bug fix: \BibTool{} has ignored \rsc{preserve.key.case} for
    cross-referenced keys. This has been fixed.
  \end{Fix}
  \begin{Fix}{gene}
    Bug fix: Some compilers don't like it if static strings are
    modified. Thus I have made them dynamic.
  \end{Fix}
  \begin{New}{gene}
    Experimental feature \rsc{regexp.syntax} added.
  \end{New}
 \end{Release}

 % =====================================================================
 \begin{Release}{2.47}{January 16, 2003}
  \begin{Fix}{gene}
    Bug fix: \BibTool{} looped when parsing unbalanced blocks due to a
    too short type in \File{parse.c}. This has been fixed and some
    error messages have been improved.
  \end{Fix}
  \begin{Doc}{gene}
    Some additions to the documentation due to Patrice Dumas.
  \end{Doc}
  \begin{New}{gene}
    Resource \rsc{fmt.word.separator} introduced. This resource can be
    used to mark additional characters to be considered as word
    separators.
  \end{New}
  \begin{Fix}{gene}
    Bug fix: \BibTool{} looped when a malformed string of some kind was
    given to expansion. This has been fixed in \verb|expand()|.
  \end{Fix}
  \begin{Fix}{gene}
    Bug fix: \BibTool{} crashed when called with empty format string
    due to an uninitialized variable. This has been fixed in
    \verb|parse_fmt()|.
  \end{Fix}
  \begin{Doc}{gene}
    Bug fix: Documentation bug fixed. The default for \rsc{sort.key}
    should be \verb|%s($key)|. This has been fixed in
    \File{Lib/default.rsc} and the manual.
  \end{Doc}
 \end{Release}

 % =====================================================================
 \begin{Release}{2.46}{February 12 2002}
  \begin{New}{gene}
    Resource \rsc{sort.cased} introduced to allow the sorting order to
    take advantage of upper and lower case characters. The default is
    false which is the previous behavior.
  \end{New}
 \end{Release}

 % =====================================================================
 \begin{Release}{2.45}{October 22 2001}
  \begin{New}{gene}
    Resource \rsc{suppress.initial.newline} introduced to suppress the
    empty line at the beginning of the output file.
  \end{New}
  \begin{Fix}{gene}
    Bug fix: Some C compilers don't like the initialization of static
    variables with \texttt{stderr}. The code has been moved to a
    initialization routine.
  \end{Fix}
  \begin{Fix}{gene}
    Bug fix: \texttt{-R} disables the reading of the resource at the
    end of processing of command line arguments.
  \end{Fix}
  \begin{Fix}{gene}
    Bug fix: Parsing of select arguments with backslashes were not treated
    correctly.
  \end{Fix}
  \begin{Doc}{gene}
    Minor documentation bugs fixed.
  \end{Doc}
 \end{Release}

 % =====================================================================
 \begin{Release}{2.44}{April 7 1999}
  \begin{Fix}{gene}
    Key generation fixed to expand \BibTeX{} strings. The current
    behavior is likely to change again in a future release.
  \end{Fix}
  \begin{Update}{gene}
    Updated to work with kpathsea 3.1 (partially).
  \end{Update}
  \begin{New}{gene}
    Searching for the kpathsea library is now the default in the
    configure script. Thus it is possible to simply place \BibTool{}
    in a web2c tree. This only requires that the directory is named
    bibtool and bibtool is added to the file lists in Makefile,
    configure, and configure.in of the web2c main directory.
  \end{New}
  \begin{Fix}{gene}
    \rsc{check.double} message improved.
  \end{Fix}
  \begin{Fix}{gene}
    Initialization bug for kpathsea fixed.
  \end{Fix}
  \begin{Fix}{gene}
    Bug fix for cross referencing: normalization to lowercase was
    missing twice.
  \end{Fix}
  \begin{Fix}{gene}
    Bug fix in the \TeX{} reading routines: spaces after active
    characters are not ignored. All characters are treated as
    unsigned. This seems to be preferable for extended ASCII
    characters.
  \end{Fix}
  \begin{Update}{gene}
    Key generation routines adapted to work with multiple databases.
    The symbol table is no longer abused to store information for the
    key generation. This may slow down the key generation a little
    bit. If it turns out to be a problem I will use a better
    algorithm.
  \end{Update}
  \begin{Update}{gene}
    Minor modifications in makefiles.
  \end{Update}
  \begin{New}{gene}
    The configuration of \BibTcl{} is now done automatically.
  \end{New}
  \begin{New}{gene}
    New data type introduced: \verb|Uchar| will hold Unicode characters in a
    future release. I just have to find regexp routines to work with UTF-8.
  \end{New}
  \begin{Update}{gene}
    Extraction extended to allow negation of the regular expression and string
    matching result. New resources \rsc{select.non} and
    \rsc{select.by.non.string} to specify such a request. C function
    \verb|add_extract()| extended to cover negated values and the former
    function \verb|add_s_extract()|.
  \end{Update}
 \end{Release}

 % =====================================================================
 \begin{Release}{2.43}{}
  \begin{New}{gene}
    Additional library \rsc{braces.rsc} to translate quotes as field
    delimiters into braces.
  \end{New}
  \begin{Update}{gene}
    \verb|bibtool -h| reports now also the library path. This can be
    used by external scripts to find the installation directory of
    \BibTool.
  \end{Update}
  \begin{Fix}{gene}
    Bug fix: comments are no longer crippled.
  \end{Fix}
  \begin{Fix}{gene}
    Minor bug fix in rewriting code: empty pattern works again.
  \end{Fix}
  \begin{Fix}{gene}
    The regression tests have revealed a bug fix in the key generation
    code. I am not aware of the change that has let to this effect but
    the new behavior is correct. Before, the macro expansion was
    suppressed under certain circumstances.
  \end{Fix}
  \begin{New}{gene}
    \BibTcl: \verb|bibtool count db| added.
  \end{New}
  \begin{Fix}{gene}
    \BibTcl: \verb|bibtool missing| ignores the case of the field
    name.
  \end{Fix}
  \begin{Update}{gene}
    \BibTcl: consistently renaming \verb|record| to \verb|entry|.
  \end{Update}
  \begin{Fix}{gene}
    Minor modifications of makefile to correctly generate
    \File{libbib.a} and install the header files.
  \end{Fix}
  \begin{Update}{gene}
    Reorganization of the code; header files are now in their own directory. I
    don't have the appropriate information for some DOS compilers yet.
    Thus minor modifications in the makefile might be required.
  \end{Update}
  \begin{New}{gene}
    C function \verb|entry_statistics()| removed. The functionality
    can be modeled with the new C function \verb|db_count()|. This
    takes into account that several databases can be managed
    internally.
  \end{New}
  \begin{Update}{gene}
    C function \verb|save_word()| renamed to \verb|add_word()|.
    \verb|list_words()| removed since it has been used for debugging
    only.
  \end{Update}
  \begin{Update}{gene}
    EntryTypes implementation changed. The API doesn't change.
    Data type \verb|StringTab| is now local to \File{symbols.c}.
    Data type \verb|Rule| is now local to \File{rewrite.c}.
  \end{Update}
  \begin{Update}{gene}
    Static library renamed from |libbib.a| to |libbibtool.a|.
  \end{Update}
  \begin{Update}{gene}
    Environment variables added to reference card.
  \end{Update}
 \end{Release}

 % =====================================================================
 \begin{Release}{2.42}{}
  \begin{New}{gene}
    New selection operation introduced. The selection with regular expressions
    is complemented by a selection according to simple sub-strings. This
    selection is initiated with the new resource \rsc{select.string}
    which takes the same arguments as the resource \rsc{select} but
    does comparison for a sub-string instead of regular expression matching.
    The value of \rsc{select.case.sensitive} is taken into account but
    at run-time and not at specification time. Certain characters can be
    ignored for the comparison. Those are contained in the new string resource
    \rsc{select.ignored}.
  \end{New}
  \begin{New}{gene}
    New resource \rsc{print.entry.types} introduced. This resource
    controls the entry types and the order in which they are printed.
    In fact the macro printing facility is obsolete with this new
    resource.
  \end{New}
  \begin{New}{gene}
    New resource \rsc{clear.ignored.words} introduced. In fact this is
    a function which deletes all ignored words. Thus it is possible to
    determine all ignored words at run-time--even the compiled in
    defaults.
  \end{New}
  \begin{Fix}{gene}
    Bug fixed: selection honors the resource \rsc{print.all.strings}
    now.
  \end{Fix}
  \begin{Update}{gene}
    Makefiles slightly improved. Less manual adaption. Installation of all
    library files -- even on Solaris.
  \end{Update}
  \begin{Fix}{gene}
    Name formatting omission fixed: Translation and length are taken
    into account correctly. The new name format sign * has been
    introduced to denote the inheritance of the translation from the
    calling format.
  \end{Fix}
  \begin{Update}{gene}
    Restriction relaxed. Now it is possible to declare up to MAXINT
    entry types and not only 4096. The \verb|Record| data structure
    and the associated macros have been modified slightly to
    accomplish this.
  \end{Update}
  \begin{Fix}{gene}
    Bug in the aux file selection mechanism fixed. This bug caused some cross
    referenced entries to be missing under special circumstances.
  \end{Fix}
  \begin{Fix}{gene}
    Bug in \File{key.c} fixed which caused a crash because an
    initialization was missing under certain circumstances.
  \end{Fix}
  \begin{New}{gene}
    Make target \File{libbib.a} added. This is known to work on UNIX. I would
    like to get feedback for other systems.
  \end{New}
  \begin{Fix}{gene}
    Minor bug in \File{names.c} fixed.
  \end{Fix}
  \begin{Fix}{gene}
    Minor bug in \File{print.c} fixed. \rsc{print.wide.equal} does not
    have exceptions any more.
  \end{Fix}
  \begin{Fix}{gene}
    Minor bug in \File{print.c} fixed. Deleted fields are treated better.
  \end{Fix}
  \begin{Fix}{gene}
    Bug in \File{database.c} fixed which caused macro expansion to fail under
    certain circumstances.
  \end{Fix}
  \begin{Update}{gene}
    \BibTcl: \verb|bibtool format| allows to use the full power of
    format specifications.
  \end{Update}
  \begin{New}{gene}
    \BibTcl: \verb|bibtool sprint| added which returns the formatted
    string representation of an entry.
  \end{New}
  \begin{Fix}{gene}
    \BibTcl: Bug in \verb|$rec set \$key ...| fixed.
  \end{Fix}
  \begin{Update}{gene}
    Minor changes in the C interface (Just wait for the compiler or
    linker to complain about different arguments:-).
  \end{Update}
 \end{Release}

 % =====================================================================
 \begin{Release}{2.41}{}
  \begin{New}{gene}
    New Boolean resource \rsc{print.equal.right} which controls whether
    the = in normal entries is aligned right or left.
  \end{New}
  \begin{New}{gene}
    New Boolean \rsc{resource print.wide.equal} which controls whether
    the equal sign is surrounded by spaces even if the alignment forces a
    narrower layout.
  \end{New}
  \begin{New}{gene}
    New Boolean resource \rsc{print.comma.at.end} which control-ls
    whether the comma between fields is printed at the end or at the beginning
    of a field/value pair.
  \end{New}
  \begin{New}{gene}
    New Boolean resource \rsc{print.deleted.entries} which controls the
    treatment of deleted entries. If this resource is true then delete entries
    are put as comments into the output. This is the old behavior and thus it
    is the default.
  \end{New}
  \begin{New}{gene}
    Pseudo field \rsc{sortkey} added.
  \end{New}
  \begin{New}{gene}
    Pseudo field \rsc{source} added which contains the file name the
    record has been read from or the empty string if this can not be
    determined.
  \end{New}
  \begin{Fix}{gene}
    Makefile adapted to meet the description.
  \end{Fix}
  \begin{New}{gene}
	configure creates \File{./config.h} as well.
  \end{New}
  \begin{New}{gene}
    Strings can be either local to a database or global. The output of macros
    does include local macros only. This is a point of incompatibility with
    previous versions.
  \end{New}
  \begin{New}{gene}
    Massive extensions to \BibTcl.
  \end{New}
  \begin{Doc}{gene}
    Cleaning of the sources and massive addition of documentation. Now the
    documentation of the C functions is present. Thus it becomes possible to
    use the \BibTool{} routines to write C programs. \BibTcl{} is a first
    application of this technique. Nevertheless a few changes seem possible
    before things are cut in stone.
  \end{Doc}
  \begin{Doc}{gene}
    According to a suggestion by Oren Patashnik the term entry is used instead
    of record -- at least in the documentation.
  \end{Doc}
  \begin{Fix}{gene}
    Minor bug in \File{print.c} fixed which caused looping when small values
    for line length and alignment column was given.
  \end{Fix}
  \begin{Fix}{gene}
    Minor bug-fix in \File{key.c}. The resource \rsc{crossref.limit} was
    off by 1.
  \end{Fix}
  \begin{New}{gene}
    Some additions have been made to support new features in the forthcoming
    \BibTeX{} 1.0. They do not really work right now but just restrict the
    accepted input files.
  \end{New}
  \begin{New}{gene}
    Flag \texttt{--with-kpathsea} added to configure.
  \end{New}
 \end{Release}

 % =====================================================================
 \begin{Release}{2.40}{}
  \begin{New}{gene}
    Sorting uses the new key instead of the old one as before. This has not
    been specified, but the new behavior might be more intuitive.
  \end{New}
  \begin{Fix}{gene}
    \rsc{key.format=empty} fixed. \rsc{key.format} \rsc{short.need}
    and \rsc{long.need} are aliases for \rsc{new.short} and
    \rsc{new.long} resp.
  \end{Fix}
  \begin{Update}{gene}
    Some error messages slightly improved.
  \end{Update}
  \begin{Fix}{gene}
    Minor print bug fixed: additional commas appeared when the last field was
    deleted.
  \end{Fix}
  \begin{Doc}{gene}
    Typo in documentation: I had typed \rsc{preserve.key} instead of
    \rsc{preserve.keys}. Shame on me:-) Some additional examples in the
    documentation.
  \end{Doc}
  \begin{Update}{gene}
    Field deletion completely covered by the rewriting mechanism.
    \rsc{delete.field} is kept for backward compatibility as an alias.
  \end{Update}
 \end{Release}

 % =====================================================================
 \begin{Release}{2.39}{}
  \begin{Update}{gene}
    The default for the pattern in the select resource is now '.'.
    Thus it is easier to select entries with a given type.
  \end{Update}
  \begin{New}{gene}
    New resource \rsc{preserve.keys} introduced. If this resource is on
    then key generation touches only entries with empty keys. The keys of
    other entries are left unchanged. Initially this resource is off.
  \end{New}
  \begin{Update}{gene}
    The disambiguation now respects the order of the entries. Previously it
    used the reverse order which was contra-intuitive.
  \end{Update}
  \begin{Update}{gene}
    Formatting names: Names connected by ~ or tight initials (A.U. Thor) are
    now treated better.
  \end{Update}
  \begin{Fix}{gene}
    Nasty little bug in the key specification parser fixed. This has
    (sometimes) let to a wrong evaluation of disjunctions.
  \end{Fix}
  \begin{Fix}{gene}
    Minor bug in verbose messages fixed.
  \end{Fix}
  \begin{New}{gene}
    AutoConf scripts added.
  \end{New}
 \end{Release}

 % =====================================================================
 \begin{Release}{2.38}{}
  \begin{New}{gene}
    Format specifier \%d enhanced. It can fail now if no number is found in
    the field. The minus sign means padding to a fixed length. The plus sign
    means not to fail but use 0. The post specifier can be used to select
    another but the first number. E.g. \%.2d means to use the second number.
  \end{New}
  \begin{New}{gene}
    Format specifier \%D introduced. It acts like \%d but does not truncate
    the number. Thus larger numbers are used completely.
  \end{New}
  \begin{Fix}{gene}
    Bug fixed. If the key format failed then the key was left empty. Now the
    default key is used as written in the documentation. Somehow the bug fix
    for the select statement has not made it into the last release. Now it
    should be in.
  \end{Fix}
  \begin{Doc}{gene}
    Several mistakes in the documentation corrected.
  \end{Doc}
 \end{Release}

 % =====================================================================
 \begin{Release}{2.37}{}
  \begin{New}{gene}
    kpathsea support for searching \BibTeX{} files added.
  \end{New}
  \begin{Fix}{gene}
    Bugs fixed: Searching/key generation combination and aux file evaluation.
  \end{Fix}
  \begin{New}{gene}
    The distribution is now available in two forms: as a gzipped tar file and
    as a zip archive.
  \end{New}
 \end{Release}

 % =====================================================================
 \begin{Release}{2.36}{}
  \begin{New}{gene}
    Added sample file for interfacing \BibTool{} with Tcl:
    \File{Tcl/bibtool.tcl}.
  \end{New}
  \begin{New}{gene}
    Added sample file for interfacing \BibTool{} with Perl:
    \File{Perl/bibtool.pl}.
  \end{New}
  \begin{Fix}{gene}
    Minor bug fix in string parsing routines. E.g. select from command line
    was not evaluated properly.
  \end{Fix}
  \begin{Fix}{gene}
    Minor bug fixed. \rsc{add.field} now works again as described in
    the documentation.
  \end{Fix}
  \begin{Fix}{gene}
    Minor bug fixed. Command line argument \Arg{-r} issued a warning
    when everything was ok.
  \end{Fix}
 \end{Release}

 % =====================================================================
 \begin{Release}{2.35}{}
  \begin{Update}{gene}
    Inheritance via crossref can be used to access fields in another
    entry when creating keys. E.g. this means that crossreferencing
    inproceedings can access the booktitle of the proceedings. The new
    resource \rsc{crossref.limit} restricts the number of crossrefs
    followed. E.g. 0 means do not follow crossrefs (as it has been in
    previous releases). The default is 32.
  \end{Update}
  \begin{New}{gene}
    New Boolean resource \rsc{check.do.delete} introduced. If this
    resource is on then the double entries are not only preceded by
    \#\#\# instead of @ but deleted completely.
  \end{New}
  \begin{New}{gene}
    New Boolean resource \rsc{sort.macros} introduced. If this is on
    then the macro definitions are sorted alphabetically. Default: on
  \end{New}
  \begin{Update}{gene}
    The handling of comments has been changed completely. Comments are
    now attached to the preamble/string/normal entry following them.
    Especially during sorting the comments are rearranged as well.
    This seems useful to keep comments on macro definitions and the
    definitions together. The last comment always stays at the end.
  \end{Update}
  \begin{Fix}{gene}
    Minor bug fixed. The searching for \File{.bibtoolrsc} went wrong.
  \end{Fix}
  \begin{Fix}{gene}
    Minor bug fixed. \rsc{output.file} resource acted incorrect.
  \end{Fix}
  \begin{New}{gene}
    \rsc{print.all.strings} acts like described in the documentation.
  \end{New}
  \begin{Update}{gene}
    Resource arguments in the command line can have unbalanced braces in
    strings now. Still there is no escape character defined.
  \end{Update}
  \begin{Update}{gene}
    Internally \BibTool{} can handle more than one database simultaneously.
    This is needed for the SQL interface. Let's see how this can be used
    otherwise.
  \end{Update}
 \end{Release}

 % =====================================================================
 \begin{Release}{2.34}{}
  \begin{Update}{gene}
    Default value of \rsc{key.expand.macros} changed to on. This seems
    to be a more sensible default value. (Possible incompatibility to old
    version)
  \end{Update}
  \begin{New}{gene}
    \rsc{new.format.type} introduced. This command can be used to
    specify formatting rules for names. It is used in conjunction with the \%p
    format specification which has been added.
  \end{New}
  \begin{New}{gene}
    The format specifiers for counting names/words/characters have been added.
    Those act as boo leans which force backtracking if they fail. The qualifier
    \# is used to enable counting.
  \end{New}
  \begin{Update}{gene}
    When writing macro files undefined macros are written as
    \verb|_string| instead of \verb|@STRING|. Thus they are
    comments to \BibTeX{}.
  \end{Update}
  \begin{Update}{gene}
    The command line options \Arg{-x} and \Arg{-X} now also set
    \rsc{print.all.strings} to off. (This has no effect yet)
  \end{Update}
  \begin{Doc}{gene}
    The documentation has been updated.
  \end{Doc}
 \end{Release}

 % =====================================================================
 \begin{Release}{2.33}{}
  \begin{New}{gene}
    Reference card added (prepared for A4 paper).
  \end{New}
  \begin{Fix}{gene}
    Bug fix in \File{rewrite.c}. \rsc{add.field} no longer crashes.
  \end{Fix}
  \begin{Update}{gene}
    Format specifiers \%n and \%N now also use the post argument. This
    argument is used to restrict the number of characters transferred.
  \end{Update}
 \end{Release}

 % =====================================================================
 \begin{Release}{2.32}{}
  \begin{New}{gene}
    New resource \rsc{select.fields} introduced. This resource
    controls which fields are considered when selecting with
    \Arg{-X}.
  \end{New}
 \end{Release}

 % =====================================================================
 \begin{Release}{2.31}{}
  \begin{Update}{gene}
    I have moved to a new job. My email address has been changed in
    all files (maybe I have missed some?).
  \end{Update}
  \begin{Fix}{gene}
    The resource \rsc{select.case.sensitive} behaved
    contra-intuitive. This has been fixed. This means that old scripts
    might need adaption.
  \end{Fix}
  \begin{Update}{gene}
    \rsc{select} and 
\rsc{rewrite.rule} can take several
    fields now. Thus you can specify several selections or rewrites
    more compact.
  \end{Update}
  \begin{Doc}{gene}
    The survey of related programs in the documentation has been
    enlarged.
  \end{Doc}
 \end{Release}

 % =====================================================================
 \begin{Release}{2.30}{}
  \begin{Fix}{gene}
    Minor bug fix in \File{s\_parse.c}. Resource specifications in
    the command line may have let to an overflow of an array. This has
    been fixed.
  \end{Fix}
  \begin{New}{gene}
    New resource \rsc{print.parentheses} introduced. This
    resource allows the user to specify whether {} or () should be
    used to delimit each entry.
  \end{New}
 \end{Release}

 % =====================================================================
 \begin{Release}{2.29}{}
  \begin{Update}{gene}
    Support for compilation without the GNU Regular Expression Library
    improved.
  \end{Update}
  \begin{Update}{gene}
    The \verb|@comment| is really more liberal than I thought. This
    prefix is simply ignored (As I have read in the \BibTeX{}
    sources). This behavior is now also performed by \BibTool{}.
  \end{Update}
 \end{Release}

 % =====================================================================
 \begin{Release}{2.28}{}
  \begin{Update}{gene}
    Some additional updates for the new makefiles.
  \end{Update}
  \begin{Fix}{gene}
    Bug fix in \File{print.c} which let to additional strings after the
    regular end of an entry.
  \end{Fix}
 \end{Release}

 % =====================================================================
 \begin{Release}{2.27}{}
  \begin{Update}{gene}
    Reorganization of the files. Slight corrections of function
    prototypes in various files. Change of the \File{Makefile} to
    support Amiga SAS/C.
  \end{Update}
 \end{Release}

 % =====================================================================
 \begin{Release}{2.26}{}
  \begin{Fix}{gene}
    Minor bug in \File{key.c} fixed. One step further to a 8-bit clean
    program. I wonder which 8-bit traps are still hidden in \BibTool.
  \end{Fix}
  \begin{Update}{gene}
    The \TeX{} macro mechanism has been enhanced to allow one to define
    single character macros. This means \rsc{tex.define}
    automatically makes a character active if it is encountered as
    macro name of a definition.
  \end{Update}
 \end{Release}

 % =====================================================================
 \begin{Release}{2.25}{}
  \begin{Update}{gene}
    Finally all separators can be arbitrary strings. Only allowed
    characters are used (as it has been done in version 1.*). I hope
    this finally fixes the bug in key.c which I had troubles with in
    the previous versions.
  \end{Update}
 \end{Release}

 % =====================================================================
 \begin{Release}{2.24}{}
  \begin{Doc}{gene}
    Documentation for regular expressions added.
  \end{Doc}
  \begin{New}{gene}
    Resources \rsc{select} and
    \rsc{select.case.sensitive} added. The selection allows to
    specify a regular expression for arbitrary fields. This subsumes
    the old \rsc{extract.regex} which is kept for backward
    compatibility.
  \end{New}
  \begin{Fix}{gene}
    A bug fix in 2.23 corrected.
  \end{Fix}
 \end{Release}

 % =====================================================================
 \begin{Release}{2.23}{}
  \begin{Update}{gene}
    Some minor modifications for DEC-Alpha.
  \end{Update}
  \begin{Update}{gene}
    Some minor modifications for the native HP C compiler.
  \end{Update}
  \begin{Fix}{gene}
    Some bugs in \File{key.c} fixed.
  \end{Fix}
 \end{Release}

 % =====================================================================
 \begin{Release}{2.22}{}
  \begin{Update}{gene}
    \BS par (double newlines) is preserved now. \rsc{parse.c}
    has been modified. The line breaking algorithm in \File{print.c}
    has been adapted.
  \end{Update}
  \begin{Fix}{gene}
    Allocation bug in \File{stack.c} fixed.
  \end{Fix}
  \begin{Fix}{gene}
    Bug in \File{expand.c} which added garbage to expanded strings.
  \end{Fix}
  \begin{Update}{gene}
    Some improvements for non-ANSI C compilers (I really don't know
    why I continue this).
  \end{Update}
 \end{Release}

 % =====================================================================
 \begin{Release}{2.21}{}
  \begin{New}{gene}
    Preserving of case of keys introduced. For this purpose the
    resource \rsc{preserve.key.case} has been introduced. The
    management of the list of old keys is done in \File{macros.c}
    mainly. Minor changes in \File{print.c} and \File{parse.c}.
  \end{New}
 \end{Release}

 % =====================================================================
 \begin{Release}{2.20}{}
  \begin{Fix}{gene}
    Minor bug fix in \rsc{sort.order}.
  \end{Fix}
  \begin{Update}{gene}
    Compression method changed to gzip.
  \end{Update}
 \end{Release}

 % =====================================================================
 \begin{Release}{2.19}{}
  \begin{New}{gene}
    Key format specifiers t, w, W introduced. w, W, t and T take a
    post argument to limit the number of characters to use.
  \end{New}
  \begin{New}{gene}
    \rsc{sort.order} introduced to sort fields in an entry.
  \end{New}
  \begin{Update}{gene}
    Wrong mail addresses in many files corrected.
  \end{Update}
 \end{Release}

 % =====================================================================
 \begin{Release}{2.18}{}
  \begin{Fix}{gene}
    Rewriting missed some multiply occurring patterns.
    (\File{rewrite.c})
  \end{Fix}
  \begin{New}{gene}
    Resource \rsc{rewrite.limit} added.
  \end{New}
  \begin{Fix}{gene}
    Problem with recursive resources fixed. (\File{parse.c})
  \end{Fix}
 \end{Release}

 % =====================================================================
 \begin{Release}{2.17}{}
  \begin{Doc}{gene}
    The documentation is now compatible with \LaTeXe.
  \end{Doc}
 \end{Release}

 % =====================================================================
 \begin{Release}{2.16}{}
  \begin{New}{gene}
    New resource \rsc{print.newline} introduced to control the
    number of empty lines between entries.
  \end{New}
  \begin{New}{gene}
    MSDOS support: Imitating emtex search path initialization as
    option.
  \end{New}
 \end{Release}

 % =====================================================================
 \begin{Release}{2.15}{}
  \begin{Fix}{gene}
    Some problems with 8-bit characters fixed.
  \end{Fix}
  \begin{New}{gene}
    String expansion before key generation added:
    \rsc{key.expand.macro}
  \end{New}
  \begin{Fix}{gene}
    Hash index was computed wrong for some strings with ASCII values $>$
    127.
  \end{Fix}
  \begin{Update}{gene}
    Hash table enlarged and new algorithm for hash index
  \end{Update}
 \end{Release}

 % =====================================================================
 \begin{Release}{2.14}{}
  \begin{Fix}{gene}
    \rsc{delete.field} bug fixed. The deletion of the last
    field lead to the deletion of all fields. (\File{rewrite.c})
  \end{Fix}
  \begin{Update}{gene}
    Comments added to \File{print.c}.
  \end{Update}
  \begin{New}{gene}
    Target zip in \File{Makefile}.
  \end{New}
 \end{Release}

 % =====================================================================
 \begin{Release}{2.13}{}
  \begin{New}{gene}
    Yet more modifications for the MSDOS port. What a brain-dead OS.
  \end{New}
 \end{Release}

 % =====================================================================
 \begin{Release}{2.12}{}
  \begin{Update}{gene}
    Version 2.12 of GNU regex library integrated.
  \end{Update}
  \begin{Update}{gene}
    \File{alloca.c} is no longer needed.
  \end{Update}
  \begin{Fix}{gene}
    Some improvements for the MSDOS port.
  \end{Fix}
  \begin{Fix}{gene}
    \File{aux.[ch]} renamed to \File{tex\_aux.[ch]} resp.
  \end{Fix}
  \begin{Fix}{gene}
    Resource search path seems to work now.
  \end{Fix}
 \end{Release}

 % =====================================================================
 \begin{Release}{2.11}{}
  \begin{Fix}{gene}
    Some typos fixed.
  \end{Fix}
  \begin{Update}{gene}
	Redefinition of entry types enabled.
  \end{Update}
 \end{Release}

 % =====================================================================
 \begin{Release}{2.10}{}
  \begin{New}{gene}
    String/Macro expansion added: \rsc{expand.macros}.
  \end{New}
  \begin{Update}{gene}
    Case in-sensitive treatment of ignored words.
  \end{Update}
  \begin{Update}{gene}
    Some additional targets in \File{Makefile}.
  \end{Update}
 \end{Release}

 % =====================================================================
 \begin{Release}{2.09}{}
  \begin{Update}{gene}
    \File{Makefile} enhanced.
  \end{Update}
  \begin{Update}{gene}
    Error/verbose/debug messages improved.
  \end{Update}
  \begin{Fix}{gene}
    Hell, I forgot to remove trace statements.
  \end{Fix}
 \end{Release}

 % =====================================================================
 \begin{Release}{2.08}{}
  \begin{Update}{gene}
    Comment parsing and printing improved.
  \end{Update}
  \begin{New}{gene}
    \rsc{symbol.type} introduced.
  \end{New}
  \begin{Update}{gene}
    Minor improvement of non-ANSI support.
  \end{Update}
 \end{Release}

 % =====================================================================
 \begin{Release}{2.07}{}
  \begin{Update}{gene}
    Printing improved. Rewriting of macro names completed and
    documented.
  \end{Update}
  \begin{Fix}{gene}
    Truncating bug in \File{print.c} fixed.
  \end{Fix}
  \begin{Update}{gene}
    Minor improvement of non-ANSI support.
  \end{Update}
 \end{Release}

 % =====================================================================
 \begin{Release}{2.06}{}
  \begin{New}{gene}
    UN*X man pages for \verb|sbuffer()|, \verb|pxfile()|
  \end{New}
  \begin{Update}{gene}
    Absolute path name use improved.
  \end{Update}
  \begin{Fix}{gene}
    Minor bug fixed.
  \end{Fix}
 \end{Release}

 % =====================================================================
 \begin{Release}{2.05}{}
  \begin{New}{gene}
    Semantic checks with regular expressions.
  \end{New}
  \begin{Update}{gene}
    Crossref modification when key generation is enabled.
  \end{Update}
  \begin{Fix}{gene}
    Preference bug for search path building fixed.
  \end{Fix}
  \begin{Update}{gene}
    Rewriting enhanced.
  \end{Update}
 \end{Release}

 % =====================================================================
 \begin{Release}{2.04}{}
  \begin{New}{gene}
    String parser introduced. Error checking/messages improved.
  \end{New}
  \begin{Update}{gene}
    Undocumented command line options changed.
  \end{Update}
  \begin{New}{gene}
    Command line specification of resources introduced.
  \end{New}
  \begin{Fix}{gene}
    Minor bug fixes.
  \end{Fix}
  \begin{Update}{gene}
    Code slightly reorganized. (I really found a static array)
  \end{Update}
  \begin{Update}{gene}
    Samples directory renamed to \File{Lib}.
  \end{Update}
 \end{Release}

 % =====================================================================
 \begin{Release}{2.03}{}
  \begin{New}{gene}
    Field rewriting introduced.
  \end{New}
  \begin{Update}{gene}
    Resource \rsc{optimize} is obsolete. The functionality can
    be achieved by field rewriting.
  \end{Update}
  \begin{New}{gene}
    Resource search path introduced.
  \end{New}
 \end{Release}

 % =====================================================================
 \begin{Release}{2.02}{}
  \begin{Fix}{gene}
    Minor bug fixes.
  \end{Fix}
  \begin{Update}{gene}
    Static array in key generation routine made dynamic. There is
    still a limit which can be reached when a short \TeX{} macro
    expands into very long text.
  \end{Update}
  \begin{New}{gene}
    Checking double entries feature introduced.
  \end{New}
  \begin{New}{gene}
    Printing TABs can be emulated by SPACEs.
  \end{New}
 \end{Release}

 % =====================================================================
 \begin{Release}{2.01}{}
  \begin{Fix}{gene}
    Bug fixes.
  \end{Fix}
  \begin{New}{gene}
    Print representation for items introduced.
  \end{New}
  \begin{Update}{gene}
    Distribution improved.
  \end{Update}
  \begin{New}{gene}
    Resource samples added.
  \end{New}
 \end{Release}

 % =====================================================================
 \begin{Release}{2.0}{}
  \begin{New}{gene}
    Major revision.
  \end{New}
  \begin{New}{gene}
    Sorting integrated.
  \end{New}
  \begin{New}{gene}
    Resources introduced.
  \end{New}
  \begin{Update}{gene}
    Several command line options deleted.
  \end{Update}
  \begin{Update}{gene}
    Key formatting improved.
  \end{Update}
  \begin{Doc}{gene}
    Draft Documentation.
  \end{Doc}
 \end{Release}

 % =====================================================================
 \begin{Release}{1.7}{}
  \item[]
 \end{Release}

\end{multicols}

\end{document}%%%%%%%%%%%%%%%%%%%%%%%%%%%%%%%%%%%%%%%%%%%%%%%%%%%%%%%%%%%%%%%%%
%
% Local Variables: 
% mode: latex
% TeX-master: nil
% End: 
