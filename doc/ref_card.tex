%%*** ref_card.tex ************************************************************
%%
%% This file is part of BibTool.
%% It is distributed under the Creative Commons Attribution-Share
%% Alike 3.0 License.
%%
%% (c) 1996-2019 Gerd Neugebauer
%%
%% Net: gene@gerd-neugebauer.de
%%
%%-----------------------------------------------------------------------------
%% Usage:  latex ref_card
%%*****************************************************************************

\documentclass[a4paper]{article}

\usepackage{multicol}
\usepackage[landscape]{geometry}
\usepackage[colorlinks=true,
            linkcolor=blue,
            urlcolor=blue,
            filecolor=blue,
            citecolor=blue]{hyperref}
%%********************************************************
%%
%% This file is part of BibTool.
%% It is distributed under the GNU General Public License.
%% See the file COPYING for details.
%%
%% (c) 1995-2019 Gerd Neugebauer
%%
%% Net: gene@gerd-neugebauer.de
%%
%%********************************************************
\newcommand\LIBDIR{/usr/local/lib/BibTool}
\newcommand\Year{2019}
\newcommand\Version{2.68}


\hypersetup{pdftitle={BibTool Quick Reference Card}}
\hypersetup{pdfauthor={Gerd Neugebauer}}
\hypersetup{pdfsubject={Version \Version}}

\oddsidemargin   = -10mm
\evensidemargin  = -10mm
\topmargin       = -10mm
\advance\textwidth  60mm
\advance\textheight 34mm
\headsep         = 0pt
\headheight      = 0pt
\parsep          = 0pt
\parindent       = 0pt
\parskip         = 0pt

\pagestyle{empty}

\newcommand\BibTool{\textsc{Bib\hskip-.1em
        T\hskip-.15emo\hskip-.05emo\hskip-.05eml}}
\newcommand\BibTeX{\textrm{B\kern-.05em\textsc{i\kern-.025em b}\kern-.08em
        T\kern-.1667em\lower.7ex\hbox{E}\kern-.125emX}}

\newenvironment{FlatList}{\begin{list}{}{%
      \topsep=0pt\itemsep=0pt\parsep=0pt\let\makelabel=\flatlistlabel}}{\end{list}}%
\newcommand\flatlistlabel[1]{\descriptionlabel{\textsf{#1}}}
\newcommand\Arg[1]{\textrm{\{}\textsl{#1}\textrm{\}}}
\newcommand\ARG[1]{\textsl{#1}}
\newcommand\OnOff{\textsl{OnOff}}
\newcommand\Num{\textsl{n}}
\newcommand\Section[1]{\begin{center}\normalsize\bfseries
    #1\end{center}\nobreak }
\newcommand\Lib[1]{\texttt{#1}}
\newcommand\mailto[1]{\href{mailto:#1}{\texttt{#1}}}
\newcommand\link[1]{\href{#1}{\texttt{#1}}}
\newcommand\NewPage{\end{multicols}
 \vfill\vfill\vfill
 \begin{center}\rule{.8\textwidth}{.1pt}\end{center}
 \newpage
 \begin{center}\rule{.8\textwidth}{.1pt}\end{center}\vfill
 \begin{multicols}{4}
}

\begin{document} %%%%%%%%%%%%%%%%%%%%%%%%%%%%%%%%%%%%%%%%%%%%%%%%%%%%%%%%%%%%%%

\begin{center}
  \Large \BibTool{} Quick Reference Card
  \\[.5ex]
  \scriptsize for \BibTool{} version \Version{}
  --- see also \link{http://www.gerd.neugebauer.de/software/TeX/BibTool/}
  \\
  \copyright\Year{} Gerd Neugebauer (\mailto{gene@gerd-neugebauer.de})
  \\
  \rule{.80\textwidth}{.1pt}
\end{center}
\vfill
\footnotesize\rmfamily
\begin{multicols}{4}
  \Section{Command line options}
  \begin{FlatList}
  \item [-{}- \ARG{rsc\_command}]\ \\
    Perform resource command as if given in a file.
  \item [-A \ARG{type}]\ \\
    Determine key disambiguation. \ARG{type} in \texttt{0}, \texttt{a}, \texttt{A},
  \item [-d\ ]\ \\
    Check double entries.
  \item [-f \ARG{key\_format}]\ \\
    Generate keys according to \ARG{key\_format}
  \item [-F\ ]\ \\
    Enable key generation with free key format.
  \item [-h\ ]\ \\
    Print short help and exit.
  \item [-i \ARG{input\_file}]\ \\
    Mark a file to be processed later.
  \item [-k\ ]\ \\
    Make keys with the short format.
  \item [-K\ ]\ \\
    Make keys with the long format.
  \item [-o \ARG{output\_file}]\ \\
    Send the output to \ARG{output\_file}.
  \item [-q\ ]\ \\
    Suppress warning messages.
  \item [-r \ARG{resource\_file}]\ \\
    Read the resource file  \ARG{resource\_file}.
  \item [-R\ ]\ \\
    Load the default resource file now.
  \item [-s\ \ ]\ \\
    Sort the result.
  \item [-S\ ]\ \\
    Sort the result in reverse order.
  \item [-v\ \ ]\ \\
    Turn on verbose messages about the actions performed.
  \item [-x \ARG{aux\_file}]\ \\
    Extract those entries mentioned in \ARG{aux\_file}.
  \item [-X \ARG{regex}]\ \\
    Extract entries matching \ARG{regex}.
  \end{FlatList}
  \columnbreak

  \Section{General}
  \begin{FlatList}
  \item [resource.search.path  	  = \Arg{dir$_1$:dir$_2$\ldots }]
  \item [resource \Arg{file}]
  \item [bibtex.search.path	  = \Arg{dir$_1$:dir$_2$\ldots }]
  \item [bibtex.env.name	  = \Arg{ENV\_NAME}]
  \item [env.separator		  = \Arg{c}]
  \item [dir.file.separator	  = \Arg{c}]
  \item [print \Arg{message}]
  \item [quiet			  = \OnOff]
  \item [verbose		  = \OnOff]
  \end{FlatList}
  \Section{Reading and Printing}
  \begin{FlatList}
  \item [input \Arg{bib\_file}]
  \item [output.file		  = \Arg{file}]
  \item [parse.exit.on.error	  = \OnOff]
  \item [pass.comments		  = \OnOff]
  \item [new.entry.type \Arg{type}]
  \item [print.align		  = \Num]
  \item [print.align.key	  = \Num]
  \item [print.align.preamble	  = \Num] 
  \item [print.align.comment	  = \Num]
  \item [print.braces		  = \OnOff]
  \item [print.comma.at.end 	  = \OnOff]
  \item [print.deleted.entries    = \OnOff]
  \item [print.deleted.prefix     = \Arg{prefix}]
  \item [print.indent		  = \Num]
  \item [print.line.length	  = \Num]
  \item [print.newline		  = \Num]
  \item [print.parentheses	  = \OnOff]
  \item [print.terminal.comma	  = \OnOff]
  \item [print.use.tab		  = \OnOff]
  \item [print.wide.equal 	  = \OnOff]
  \item [suppress.initial.newline = \OnOff]
  \item [new.field.type \Arg{new=old}]
  \item [symbol.type		  = \ARG{type}]\ \\
    upper, lower, cased
  \end{FlatList}
  \columnbreak
  \Section{Sorting}
  \begin{FlatList}
  \item [sort = \OnOff]
  \item [sort.cased = \OnOff]
  \item [sort.reverse = \OnOff]
  \item [sort.format = \Arg{format}]
  \item [sort.order \Arg{\ldots }]
  \item [sort.macros = \OnOff]
  \end{FlatList}
  \Section{Searching (Extraction)}
  \begin{FlatList}
  \item [tex.define \Arg{macro[arg]=text}]
  \item [extract.file \Arg{file}]
  \item [select \Arg{field$_1$\ldots field$_n$ "regex"}]
  \item [select \Arg{type$_1$\ldots type$_n$ }]
  \item [select.by.string \Arg{field$_1$\ldots field$_n$ "regex"}]
  \item [select.by.string.ignore \Arg{chars}]
  \item [select.case.sensitive = \OnOff]
  \item [select.fields = \Arg{field$_1$,field$_2$,\ldots }]
  \end{FlatList}
  \Section{Field Manipulation}
  \begin{FlatList}
  \item [add.field \Arg{field="value"}]
  \item [delete.field \Arg{field}]
  \item [keep.field \Arg{field}]
  \item [keep.field \Arg{field if field$_2$="pattern"}]
  \item [rename.field \Arg{old=new}]
  \item [rename.field \Arg{old=new if field="pattern"}]
  \item [rewrite.rule \Arg{ pattern }]\ \\
    delete all matching fields
  \item [rewrite.rule \Arg{ pattern \# replacement}]\ \\
    rewrite all fields
  \item [rewrite.rule \Arg{f$_1$\ldots f$_n$ \# pattern \# replacement}]\ \\
    rewrite some fields
  \item [rewrite.case.sensitive = \OnOff]
  \item [rewrite.limit = \Arg{n}]
  \end{FlatList}
  \columnbreak
  \Section{Checks}
  \begin{FlatList}
  \item [check.double = \OnOff]
  \item [check.do.delete = \OnOff]
  \item [check.rule \Arg{field \# pattern \# message}]
  \item [check.warning.rule \Arg{field \# pattern \# message}]
  \item [check.error.rule \Arg{field \# pattern \# message}]
  \item [check.case.sensitive = \OnOff]
  \item [unique.field \Arg{field}]
  \end{FlatList}
  \Section{Strings}
  \begin{FlatList}
  \item [macro.file \Arg{file}]
  \item [print.all.strings = \OnOff]
  \item [expand.macros = \OnOff]
  \end{FlatList}
  \Section{Inheritance}
  \begin{FlatList}
  \item [crossref.map = \OnOff]
  \item [clear.crossref.map \Arg{}]
  \item [crossref.limit = \Arg{n}]
  \item [expand.crossref = \OnOff]
  \item [expand.xdata = \OnOff]
  \end{FlatList}
  \Section{\BibTeX1.0}
  \begin{FlatList}
  \item [apply.alias = \OnOff]
  \item [apply.include = \OnOff]
  \item [apply.modify = \OnOff]
  \item [key.make.alias = \OnOff]
  \end{FlatList}
  \Section{Counting}
  \begin{FlatList}
  \item [count.all = \OnOff]
  \item [count.used = \OnOff]
  \end{FlatList}
\NewPage
  \Section{Key Generation}
  \begin{FlatList}
  \item [preserve.keys	 	= \OnOff]
  \item [preserve.key.case	= \OnOff]
  \item [key.format		= \Arg{format}]\ \\
    special values: short, long, short.need, long.need, empty
  \item [key.generation		= \OnOff]
  \item [default.key		= \Arg{key}]
  \item [key.base		= \ARG{base}]\ \\
    values: upper, lower, digit
  \item [key.number.separator	= \Arg{s}]
  \item [key.expand.macros	= \OnOff]
  \item [fmt.name.title		= \Arg{s}]
  \item [fmt.title.title	= \Arg{s}]
  \item [fmt.name.name		= \Arg{s}]
  \item [fmt.inter.name		= \Arg{s}]
  \item [fmt.name.pre		= \Arg{s}]
  \item [fmt.et.al		= \Arg{s}]
  \item [fmt.word.separator	= \Arg{s}]
  \item [new.format.type	= \Arg{n="spec"}]
  \end{FlatList}
  \Section{Name Formatting Specification}
    Use \textit{n}\/ letters. Use \textit{m} name parts.  Insert \textit{pre}\/
    before, \textit{mid} between, and \textit{post} after the words.
    Translate according to the s parameter ('+', '-', '*').
  \begin{FlatList}
  \item [\%\textit{s}\textit{n}.\textit{m}\/f{[\textit{mid}][\textit{pre}][\textit{post}]}]\ \\
    format first names. 
  \item [\%\textit{s}\textit{n}.\textit{m}\/v{[\textit{mid}][\textit{pre}][\textit{post}]}]\ \\
    format ``von'' part.
  \item [\%\textit{s}\textit{n}.\textit{m}\/l{[\textit{mid}][\textit{pre}][\textit{post}]}]\ \\
    format last name.
  \item [\%\textit{s}\textit{n}.\textit{m}\/j{[\textit{mid}][\textit{pre}][\textit{post}]}]\ \\
    format ``junior'' part.
  \end{FlatList}
  \Section{Format Specifications}
  \textbf{Pseudo fields:} 
  \begin{FlatList}
  \item [\$key]
  \item [\$default.key]
  \item [\$sortkey]
  \item [\$source]
  \item [\$type]
  \item [@type]
  \item [\$day]
  \item [\$month]
  \item [\$mon]
  \item [\$year]
  \item [\$hour]
  \item [\$minute]
  \item [\$second]
  \item [\$user]
  \item [\$hostname]
  \end{FlatList}
  \textbf{Formatting Fields:} 
  \begin{FlatList}
  \item [\%$\pm $\textit{x}.\textit{y}\/ n(\textit{field}\/)]\ \\
    format \textit{y}\/ characters of \textit{x}\/ last names.
  \item [\%$\pm $\textit{x}.\textit{y}\/ N(\textit{field}\/)]\ \\
    format \textit{y}\/ characters of \textit{x}\/ names.
  \item [\%$\pm $\textit{x}.\textit{y}\/ p(\textit{field}\/)]\ \\
    format \textit{x}\/ names according to the name format \textit{y}.
  \item [\%$\pm $\textit{x}.\textit{y}\/ d(\textit{field}\/)]\ \\
    format at most \textit{x}\/ digits of the \textit{y}$^{th}$ number.
  \item [\%$\pm $\textit{x}.\textit{y}\/ D(\textit{field}\/)]\ \\
    format \textit{x}\/ digits of the \textit{y}$^{th}$ number without truncation.
  \item [\%$\pm $\textit{x}\/ s(\textit{field}\/)]\ \\
    format \textit{x}\/ string characters.
  \item [\%$\pm $\textit{x}.\textit{y}\/ t(\textit{field}\/)]\ \\
    format \textit{x}\/ sentence words of length \textit{y}.
  \item [\%$\pm $\textit{x}.\textit{y}\/ T(\textit{field}\/)]\ \\
    format \textit{x}\/ sentence words  of length \textit{y}. (Words ignored)
  \item [\%$\pm $\textit{x}.\textit{y}\/ w(\textit{field}\/)]\ \\
    format \textit{x}\/ words of length \textit{y}.
  \item [\%$\pm $\textit{x}\/ W(\textit{field}\/)]\ \\
    format \textit{x}\/ words of length \textit{y}. (Words ignored)
  \item [\%$\pm $\textit{x}.\textit{y}\/ \#n(\textit{field}\/)]\ \\
    test whether the  number of names is between \textit{x}\/ and \textit{y}.
  \item [\%$\pm $\textit{x}.\textit{y}\/ \#N(\textit{field}\/)]\ \\
    test whether the  number of names is between \textit{x}\/ and \textit{y}.
  \item [\%$\pm $\textit{x}.\textit{y}\/ \#p(\textit{field}\/)]\ \\
    test whether the  number of names is between \textit{x}\/ and \textit{y}.
  \item [\%$\pm $\textit{x}.\textit{y}\/ \#s(\textit{field}\/)]\ \\
    test whether the  number of characters is between \textit{x}\/ and \textit{y}.
  \item [\%$\pm $\textit{x}.\textit{y}\/ \#t(\textit{field}\/)]\ \\
    test whether the  number of words is between \textit{x}\/ and \textit{y}.
  \item [\%$\pm $\textit{x}.\textit{y}\/ \#T(\textit{field}\/)]\ \\
    test whether the  number of not ignored words is between \textit{x}\/
    and \textit{y}. 
  \item [\%$\pm $\textit{x}.\textit{y}\/ \#w(\textit{field}\/)]\ \\
    test whether the  number of words is between \textit{x}\/ and \textit{y}.
  \item [\%$\pm $\textit{x}.\textit{y}\/ \#W(\textit{field}\/)]\ \\
    test whether the  number of not ignored words is between \textit{x}\/
    and \textit{y}. 
  \end{FlatList} 

  \columnbreak
  \Section{Libraries}
  \begin{tabular}{lp{.17\textwidth}}
  \Lib{check\_y}	& Check the value of the year.\\
  \Lib{default}		& All default settings.\\
  \Lib{field}		& Redefine field names.\\
  \Lib{brace}		& Use braces as delimiters.\\
  \Lib{improve}		& Apply improvements.\\
  \Lib{iso2tex}		& Translate ISO\,8859/1 characters.\\
  \Lib{iso\_def}	& Define ISO\,8859/1 characters for formatting.\\
  \Lib{keep\_bibtex}	& Keep only the fields of standard \BibTeX\ styles.\\
  \Lib{keep\_biblatex}	& Keep only the fields of standard bib\LaTeX\ styles.\\
  \Lib{month}		& Introduce strings for month names.\\
  \Lib{opt}		& Remove OPT in field names.\\
  \Lib{sort\_fld}	& Specify sort order for fields.\\
  \Lib{tex\_def}	& Define \TeX{} macros for formatting.\\
  \Lib{biblatex}	& Define entry types and fields known to bib\-\LaTeX.\\
  \end{tabular}

  \Section{Environment Variables}
  \begin{tabular}{lp{.17\textwidth}}
  \texttt{BIBTOOL}	& Search path for rsc files\\
  \texttt{BIBINPUTS}	& Search path for bib files\\
  \texttt{HOME}		& Home directory for \texttt{.bibtoolrsc}\\
  \end{tabular}

\end{multicols}
\vfill\vfill\vfill
\begin{center}\rule{.8\textwidth}{.1pt}\\
Source repository and issue tracker at \link{https://github.com/ge-ne/bibtool}
\end{center}
\end{document} %%%%%%%%%%%%%%%%%%%%%%%%%%%%%%%%%%%%%%%%%%%%%%%%%%%%%%%%%%%%%%%%
%
% Local Variables:
% mode: latex
% TeX-master: nil
% End:
